\documentclass[12pt]{article}
\usepackage[utf8]{inputenc}
\usepackage{float}
\usepackage{amsmath}


\usepackage[hmargin=3cm,vmargin=6.0cm]{geometry}
%\topmargin=0cm
\topmargin=-2cm
\addtolength{\textheight}{6.5cm}
\addtolength{\textwidth}{2.0cm}
%\setlength{\leftmargin}{-5cm}
\setlength{\oddsidemargin}{0.0cm}
\setlength{\evensidemargin}{0.0cm}

%misc libraries goes here
%\usepackage{fitch}


\begin{document}

\section*{Student Information }
%Write your full name and id number between the colon and newline
%Put one empty space character after colon and before newline
Full Name :  Adil Kaan Akan \\
Id Number :  2171155 \\

% Write your answers below the section tags
\section*{Answer 1}
\begin{table}[H]
\small
\centering
\caption{ Truth Table for Question1.1 }
\label{table:example}
\begin{tabular}{|c|c|c|c|c|c|c|}	%% specify column number
\hline 							%% line draw
\textbf{$p$} & \textbf{$q$} & \textbf{$\neg p$} & \textbf{$\neg q$} & \textbf{$p \to q$} & \textbf{$\neg q \land (p \to q$} & \textbf{$(\neg q \land (p \to q) \to \neg p$}\\
\hline
\hline
T & T & F & F & T & F & T \\			%% rows distinguished with &
T & F & F & T & F & F & T \\
F & T & T & F & T & F &  T \\
F & F & T & T  & T & T & T \\
\hline

\end{tabular}
\end{table}

\begin{table}[H]
\small
\centering
\caption{ Truth Table for Question1.2 }
\label{table:example}
\begin{tabular}{|c|c|c|c|c|c|c|c|c|}	%% specify column number
\hline 							%% line draw
\textbf{$p$} & \textbf{$q$} & \textbf{$r$} & \textbf{$\neg p$} & \textbf{$p \lor q$}  & \textbf{$(\neg p \lor r)$}& \textbf($q \lor r$) & \textbf{$(p \lor q) \land (\neg p \lor r)$} & \textbf{$((p \lor q) \land (\neg p \lor r)) \to (q \lor r)$}\\
\hline
\hline
T & T & T & F & T & T & T & T & T\\			%% rows distinguished with &
T & T & F & F & T & F & T & F & T \\
T & F & T & F & T & T & T & T & T\\
T & F & F & F & T & F & F & F & T\\
F & T & T & T & T & T & T & T & T\\
F & T & F & T & T & T & T & T & T\\
F & F & T & T & T & T & T & F & T\\
F & F & F & T & F & T & F & F & T\\
\hline

\end{tabular}
\end{table}

\section*{Answer 2}
$(p \to q) \lor (p \to r) \equiv (\neg q \land \neg r) \to \neg p$ \textit{Given} \\

$(p \to q) \lor (p \to r) \equiv (\neg p \lor q) \lor (\neg p \lor r)$  \textbf{Table 7} \\

$(\neg p \lor q) \lor (\neg p \lor r) \equiv (\neg p \lor \neg p) \lor (q \lor r)$ \textbf{Associative law}\\

$(\neg p \lor \neg p) \lor (q \lor r) \equiv \neg p \lor (q \lor r)$ \textbf{Idempotent law} \\

$\neg p \lor (q \lor r) \equiv q \lor r \lor \neg p$ \textbf{Commutative law} \\

$(\neg q \land \neg r) \to \neg p \equiv \neg(\neg q \land \neg r) \lor \neg p$ \textbf{Table 7}\\

$\neg(\neg q \land \neg r) \lor \neg p \equiv q \lor r \lor \neg p$ \textbf{De Morgan law} \\

$q \lor r \lor \neg p \equiv q \lor r \lor \neg p$ \textit{Result}\\

\section*{Answer 3}
1. \\
a)Every cat has at least one friend which is dog.\\
b)There exists a cat that is friends with all dogs. \\
2.\\
a)$\forall_x\forall_y(Meal(y) \land Eats(x,y) \to Customer(x))$ \\
b)$\neg \exists_x(Chef(x) \to \forall_y(Meal(y) \to Cooks(x,y))$ \\
c)$\exists_x\forall_y\exists_z(Customer(x) \land Chef(z) \land Meal(y) \land Cooks(z,y) \land Eats(x,y) \land \forall_t(chef(t) \land Cooks(t,y) \to t = z))$ \\
d)$\forall_x\exists_y(Chef(x) \to Chef(y) \land Knows(x,y) \land \forall_z(Meal(z) \land Cooks(y,z) \to \neg Cooks(x,z)))$



\section*{Answer 4}
\begin{table}[H]
\small
\caption{Truth Table for Deduction rule}
\centering
\label{table3}
\begin{tabular}{|c|c|c|c|c|}	%% specify column number
\hline 							%% line draw
\textbf{p} & \textbf{q} & \textbf{$\neg p $} & \textbf{$p \rightarrow q$} & \textbf{$ \neg q$}  \\
\hline
\hline
T & T & F & T & F \\			%% rows distinguished with &
T & F & F & F & T \\
F & T & T & T & F \\
F & F & T & T & T \\
\hline
\end{tabular}
\end{table}

Natural deduction says that if we have premises that are true, then the right hand side 	proposition will also be true. However, when we build a truth table, in the cases where premises are true, we do not have true right hand side proposition. Therefore, this system cannot be deduction rule.

\section*{Answer 5}
\begin{table}[H]
	\centering
	\caption{Answer to the 5th question }
	\begin{tabular}{*6{l}}
		$1$ & & & $p \to q$ & \textit{premise} & \\
		$2$ & & & $q \to r$ & \textit{premise} & \\

		$3$ & & & $r \to p$ &\textit{premise} &\\

		\hline

		$4$ & & & $p$ & \textit{assumption} &\\

		$5$ & & & $q$ & $\rightarrow_e,1$ &\\

		$6$ & & & $r$ & $\rightarrow_e,2$ &\\

		\hline

		$7$ & & & $p \to r$ & $\rightarrow_i,4-6$ &\\


		$8$ & & & $p \leftrightarrow r$ & $\leftrightarrow_i,3,7$ &\\

		\hline

		$9$ & & & $q$ & \textit{assumption} &\\

		$10$ & & & $r$ & $\rightarrow_e,2$ &\\

		$11$ & & & $p$ & $\rightarrow_e,3$ &\\

		\hline

		$12$ & & & $q \to p$ & $\rightarrow_i,9-11$ &\\

		$13$ & & & $p \leftrightarrow q$ &$\leftrightarrow_i,1,12$ &\\

		$14$ & & & $(p \leftrightarrow r) \land (p \leftrightarrow q)$ & $\land_i,5,8$ &\\




	\end{tabular}
\end{table}

\section*{Answer 6}
\begin{table}[H]
	\centering
	\caption{Answer to the 6th question }
	\begin{tabular}{*6{l}}
		$1$ & & & $\forall(Q(x) \to P(x))$ & \textit{premise} & \\
		$2$ & & & $\exists(P(x) \to Q(x))$ & \textit{premise} & \\

		$3$ & & & $\forall P(x)$ &\textit{premise} &\\

		$4$ & & & $Q(x) \to P(x)$ & $\forall_e,1$ &\\

		

		\hline

		$5$ &  &  & $P(c) \to Q(c)$ &\textit{assumption} &\\
		
		$6$ & & & $P(c)$ & $\forall_e,3$ &\\


		$7$ & & & $Q(c)$ &$\rightarrow_e,5,6$ &\\

		$8$ & & & $R(c)$ & $\rightarrow_e,4,7$ &\\

		$9$ & & & $P(c) \land R(c)$ & $\land_i,5,8$ &\\



		$10$ & & & $\exists(P(x) \land R(x))$ & $\exists _i,9$ & \\

		\hline

		$11$ & & & $\exists(P(x) \land R(x))$ & $\exists _e,2,6-10$ & \\


	\end{tabular}
\end{table}


\end{document}
