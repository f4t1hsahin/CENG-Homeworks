\documentclass[12pt]{article}
\usepackage[utf8]{inputenc}
\usepackage{float}
\usepackage{amsmath}


\usepackage[hmargin=3cm,vmargin=6.0cm]{geometry}
%\topmargin=0cm
\topmargin=-2cm
\addtolength{\textheight}{6.5cm}
\addtolength{\textwidth}{2.0cm}
%\setlength{\leftmargin}{-5cm}
\setlength{\oddsidemargin}{0.0cm}
\setlength{\evensidemargin}{0.0cm}



\begin{document}

\section*{Student Information } 
%Write your full name and id number between the colon and newline
%Put one empty space character after colon and before newline
Full Name :  Adil Kaan Akan\\
Id Number :  2171155\\

% Write your answers below the section tags
\section*{Answer 1}
Basis Step: \\
For n = 1, we have
$(1)^2 \geq 1^2$ is true. \\
Inductive Step: \\
We need to show that if P(x) is true, P(x+1) is true. \\
For P(x+1), we have \\
$(\sum_{k = 1}^{x+1} k)^2 \geq \sum_{k=1}^{x+1}{k^2}$ \\
Since that is summation symbol, we can seperate $(x+1)^{th}$ terms. \\
$(\sum_{k = 1}^{x} k + (x+1))^2 \geq \sum_{k=1}^{x}{k^2 + x^2}$ \\
If we take square of left side \\
$(\sum_{k = 1}^{x} k)^2 + 2*(x+1)*(\sum_{k = 1}^{x} k)+(x+1)^2 \geq \sum_{k=1}^{x}{k^2 + x^2}$ \\
We now $(\sum_{k = 1}^{x+1} k)^2 \geq \sum_{k=1}^{x+1}{k^2}$ term by P(x) is true and we can omit it.
$2*(x+1)*(\sum_{k = 1}^{x} k)+(x+1)^2 \geq  x^2$ \\
$2*(x+1)*(\sum_{k = 1}^{x} k)+x^2 + 2x + 1  \geq  x^2$ \\
$2*(x+1)*(\sum_{k = 1}^{x} k) + 2x + 1  \geq  0$ \\
$2*(x+1)*(\sum_{k = 1}^{x} k)$ we know this term is greater then zero and $2x+1$ is also greater than zero for $x \geq 1$. Then, we prove the hypothesis.
We know that P(x+1) is true, when P(x) is true. Inductive hypothesis states that $(\sum_{k = 1}^{x+1} k)^2 \geq \sum_{k=1}^{x+1}{k^2}$ is true for all integers greater than 1.


\section*{Answer 2}
\subsection*{2.1}
There exits 20 pairs such that their sum is 42.The pairs are \\
$(1,41),(2,40),(3,39),......,(19,23), (20,22)$ \\
When $22^{nd}$ integer is selected, the game will be over because there will 21 integer in worse case such that $1,2,3,4,5,6,....,21$ and except 21 others are element of one of the pairs. If we think integers as pigeon and pairs pigeonholes, when $22^{nd}$ integer is selected, we have 20 pigeons in pigeonholes and $22^{nd}$ will be $21^{st}$ pigeon. Then, we have 21 pigeons and 20 pigeonholes, The pigeonhole principle states that we have at least 2 pigeons in one pigeonhole. That is, $22^{nd}$ integer will be element of the one of the pairs and game will be over and Bob will lose the game.
\subsection*{2.2}
The question asks how we can get 5 from 3 integers. That is, \\
(5,0,0) \\
(4,1,0) \\
(3,1,1) \\
(3,2,0) \\
(2,2,1) \\
When order is not important we do not need permutations of these triplets and therefore, answer is 5.
\subsection*{2.3}
When order important and we need 3 positive integers. We need $x_1 \geq 0$ and $x_2 \geq 0$ and $x_3 \geq 0$ \\
If we distribute 3 ones, then we have $x_1+x_2+x_3 = 2$ \\
We can distribute this with introducing 2 seperators.(Think dots(.) as numbers and slashs(/) as seperators). We have \\
$..//$ then we can distribute this with 2 combination of 4. Then we have the answers that is 6.


\section*{Answer 3}
$(1-x^3)^n = \sum_{k=0}^na_kx^k(1-x)^{3n-2k}$ \\
If we divide both sides $(1-x)^n$. Then, we get \\
$(x^2+x+1)^n = \sum_{k=0}^na_kx^k(1-x)^{2n-2k}$ \\
If we arrange both sides \\
$((1-x)^2 + 3x)^n = \sum_{k=0}^na_kx^k((1-x)^2)^{n-k}$
If we use binomial expansion to left hand side, we get\\
$\sum_{k=0}^n\binom{n}{k}((1-x)^2)^{n-k}3^kx^k = \sum_{k=0}^na_kx^k((1-x)^2)^{n-k}$ \\
Inside of the summation symbols are equal. Then, we have \\
$\binom{n}{k}((1-x)^2)^{n-k}3^kx^k = a_kx^k((1-x)^2)^{n-k}$ \\
It is obvious that $a_k = \binom{n}{k}3^k$ \\


\section*{Answer 4}
We need particular and homogenous solution.For homogenous solution, we need to look at charactheristic equation, which is \\
$r^3-4r^2+r+6 = 0$ \\
If we solve the equation, the roots are $r_1 = -1, r_2 = 2, r_3 = 3$. Then, we have \\
$a^{(h)} = c_1(-1)^n+c_2(2)^n+c_3(3)^n$ \\
For particular solution, we assume that $a^{(p)} = cn + d$, and solve with equation $a_n - 4a_{n-1} + a_{n_2} + 6a_{n-3} = n- 2$ \\
If we substitude $a^{(p)}$, we will get \\
$cn+d - 4*(c(n-1)+d) + c(n-2) + d + 6(c(n-3) + d) = n-2$ \\
We will get $c = 1/4$ and $d = 1/2$ \\
$a = a^{(h)} + a^{(p)}$ \\
$a = c_1(-1)^n + c_2(2)^n + c_3(3)^n + n/4 + 1/2$ \\
$a_0 = c_1 + c_2 + c_3  + 1/2 = 3.5$ \\
$a_1 = -c_1 + 2c_2 + 3c_3 + 1/2+ 1/2 = 4.75$ \\
$a_2 = c_1 + 4c_2 + 9c_3+1+1/2 = 13$ \\
If we solve the equations, we get \\
$c_1 = 5/6 , c_2 = 5/3, c_3 = 1/2$ \\
Then, the solution is \\
$a_n = 5/6 * (-1)^n + 5/3*(2)^n + 1/2*(3)^n + n/4 + 1/2$ \\

\end{document}

​

