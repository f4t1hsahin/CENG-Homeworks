\documentclass[11pt]{article}
\usepackage[utf8]{inputenc}
\usepackage{float}
\usepackage{amsmath}


\usepackage[hmargin=3cm,vmargin=6.0cm]{geometry}
%\topmargin=0cm
\topmargin=-2cm
\addtolength{\textheight}{6.5cm}
\addtolength{\textwidth}{2.0cm}
%\setlength{\leftmargin}{-5cm}
\setlength{\oddsidemargin}{0.0cm}
\setlength{\evensidemargin}{0.0cm}

\newcommand{\HRule}{\rule{\linewidth}{1mm}}

%misc libraries goes here
\usepackage{tikz}
\usetikzlibrary{automata,positioning}

\begin{document}
\noindent
\HRule
\begin{center}
\Large 
\textbf{CENG 222}  \\
\normalsize 
Assignment 2 \\
Deadline: May 13, 23:59 \\
\end{center}
\begin{flushleft}
\normalsize 
	Full Name: Adil Kaan Akan\\
	Id Number: 2171155
\end{flushleft}
\HRule

% Write your answers below the section tags
\section*{Answer 9.16}
\subsection*{a}
Since there are a lot sample, we can use the values in the question directly.\\
$\hat{a} = \frac{10}{250}= 0.04$\\
$\hat{b} = \frac{18}{300} = 0.06$ \\
$z_{\alpha/2} = 2.326$ \\
If we use the formula for difference,\\
\begin{align*}
& =  (\hat{p}_A - \hat{p}_B) \pm Z_{\alpha/2}\sqrt{\frac{\hat{p}_A (1-\hat{p}_A)}{n_A} + \frac{\hat{p}_B (1-\hat{p}_B)}{n_B}} \\
&= (0.04 - 0.06) \pm 2.326\sqrt{ \frac{0.04(1-0.04)}{250} + \frac{0.06(1-0.06)}{300}}\\
&= -0.02 \pm 0.043\\
&=(-0.063,0.023)
\end{align*}
\subsection*{b}
The level of significance, $\alpha = 0.02$ \\
\begin{align*}
Z &= \frac{\hat{p}_A - \hat{p}_B}{\sqrt{\frac{\hat{p}_A (1-\hat{p}_A)}{n_A} + \frac{\hat{p}_B (1-\hat{p}_B)}{n_B}}} \\
&= \frac{0.04 - 0.06}{\sqrt{ \frac{0.04(1-0.04)}{250} + \frac{0.06(1-0.06)}{300}}} \\
&= -1.06
\end{align*}
We should find the p-value for those two lots. \\
\begin{align*}
\textit{p-value} &= 2P(Z < Z_0) \\
&= 2x P (Z \leq -1.06) \\
&= 2 x 0.14457 \\
&= 0.28915 \\
\end{align*}
Since we found a p-value that is greater than our significance level after doing our calculations, we can say that there is no significant difference between the qualites of two lots.
\section*{Answer 10.2}
When we calculate the mean of the 64 observations of the system, we get $\bar{X} = 5.0$ \\
The exponential distribution says that the Cumulative distribution function F(X),\\
\begin{align*}
F(x) &= P(X \leq x) \\
&= 1 - e^{-\lambda x} \textit{     where } 0 \leq x \leq \infty \\\
\lambda &= \frac{1}{\bar{X}} \\
&= \frac{1}{5.0} \\
&= 0.2
\\
\\
F(x) &= 1 - e^{-0.2x} \textit{     where } 0 \leq x \leq \infty
\end{align*}
By using that function we can calculate the expected fequencies of $j^{th}$ class $(a_j - b_j)$ by using \\
$e_j = F(b_j) - F(a_j)$ \\
After calculating it, we should test the assumption at 5\% significance level whether the asssumption of the Exponentiality supported by these data. \\
We can test it by using the following function,\\
$X^2 = \sum (\frac{(o_j - e_ j)^2}{e_j})$ where\\ $o_j$ is the number of frequencies in the intervals(0-2,2-4,4-6,....,14-16)\\
$e_j = F(b_j) - F(a_j)$ \\
\begin{table}[H]
\centering
\begin{tabular}{|l|l|l|l|}
\hline
Class Interval & o\_j & e\_j  & (o\_j - e\_j)\textasciicircum{}2 / e\_j \\ \hline
0-2            & 13   & 21.10 & 3.11                                    \\ \hline
2-4            & 16   & 14.14 & 0.24                                    \\ \hline
4-6            & 15   & 9.48  & 3.21                                    \\ \hline
6-8            & 7    & 6.36  & 0.07                                    \\ \hline
8-10           & 5    & 4.26  & 0.13                                    \\ \hline
10-12          & 5    & 2.86  & 1.61                                    \\ \hline
12-14          & 2    & 1.91  & 0.00                                    \\ \hline
14-16          & 1    & 3.89  & 2.15                                    \\ \hline
Total          & 64   & 64    & 10.52                                   \\ \hline
\end{tabular}
\end{table}

Since there are values that are less than 5 in $e_j$ column, we should merge them. \\

\begin{table}[H]
\centering
\begin{tabular}{|l|l|l|l|}
\hline
Class Interval & o\_j & e\_j  & (o\_j - e\_j)\textasciicircum{}2 / e\_j \\ \hline
0-2            & 13   & 21.10 & 3.11                                    \\ \hline
2-4            & 16   & 14.14 & 0.24                                    \\ \hline
4-6            & 15   & 9.48  & 3.21                                    \\ \hline
6-8            & 7    & 6.36  & 0.07                                    \\ \hline
8-12           & 10    & 7.12  & 1.16                                    \\ \hline
12-16          & 3    & 5.80  & 1.35                                    \\ \hline
Total          & 64   & 64   & 9.14                                   \\ \hline
\end{tabular}
\end{table}

\begin{align*}
X^2 &= \sum (\frac{(o_j - e_ j)^2}{e_j}) \\
&= 9.14
\end{align*}
The degrees of freedom in this case is\\
\begin{align*}
df &= n-1 - 1\\
&=6 - 1 - 1\\
&=4
\end{align*}
The conclusion:\\
If we look to the p-value for the test at 4 df, our value is 9.14 and the $\alpha$ value is in between 0.05 and 0.1. In generally $\alpha$ values are in that interval. We can say that there is not sufficient evidence to reject the assumption. So, we should accept the assumption of Exponentiality is not supported by these data.

\section*{Answer 10.3}
\subsection*{a}
When we calculate the mean and the standard deviation of the 100 observations, we get $\bar{X} = -0.058$ and $\sigma = 1.058$.\\


\begin{table}[H]
\centering
\begin{tabular}{|l|l|l|l|}
\hline
Class Size    & o\_j & e\_j  & (o\_j-e\_j)\textasciicircum{}2/e\_j \\ \hline
below -1.5    & 8    & 6.68  & 0.26                                \\ \hline
-1.5 to -1.0  & 15   & 9.19 & 3.67                                \\ \hline
-1.0 to -0.5  & 9    & 14.98 & 2.39                                \\ \hline
-0.5 to 0.0   & 22   & 19.15 & 0.42                                \\ \hline
0.0 to 0.5    & 15   & 22.21 & 2.34                                \\ \hline
0.5 to 1.0    & 12   & 13.97 & 0.28                                \\ \hline
1.0 to 1.5    & 11   & 8.2  & 0.96                                \\ \hline
1.5 and above    & 8    & 5.59  & 1.04                                \\ \hline
Total         & 100  & 100   & 11.36                              \\ \hline
\end{tabular}
\end{table}

We should test the assumption 5 \% significance level whether the data follows the normal distribution.\\
\begin{align*}
X^2 &= \sum \frac{(o_j - e_j)^2}{e_j}\\
&= 11.36
\end{align*}

Degrees of freedom \\
\begin{align*}
df &= n-1\\
&=8 - 1\\
&= 7
\end{align*}
If we look to the p-value for the test at 7 df, our value is 11.36 and the $\alpha$ value is in between 0.1 and 0.2. Since our value is greater than general $\alpha$ interval(0.05, 0.1), we should accept the assumption whether the data follows the standard normal distribution.
\subsection*{b}
The pdf of the uniform distribution is \\
$f(x) = \frac{1}{b-a}$ where $a \leq x \leq b$\\
For this question, a is -3 and b is 3.\\

\begin{table}[H]
\centering
\begin{tabular}{|l|l|l|l|}
\hline
Class Size    & o\_j & e\_j  & (o\_j-e\_j)\textasciicircum{}2/e\_j \\ \hline
below -1.5    & 8    & 25  & 11.56                                \\ 
\hline
-1.5 to -1.0  & 15   & 8.33 & 5.34                                \\ \hline
-1.0 to -0.5  & 9    & 8.33 & 0.05                                \\ \hline
-0.5 to 0.0   & 22   & 8.33 & 22.43                                \\ \hline
0.0 to 0.5    & 15   & 8.33 & 5.34                                \\ \hline
0.5 to 1.0    & 12   & 8.33 & 1.62                                \\ \hline
1.0 to 1.5    & 11   & 8.33  & 0.86                                \\ \hline
1.5 and above    & 8    & 25  & 11.56                                \\ \hline
Total         & 100  & 100   & 58.76                              \\ \hline
\end{tabular}
\end{table}

We should test the assumption 5 \% significance level whether the data follows the uniform distribution.\\
\begin{align*}
X^2 &= \sum \frac{(o_j - e_j)^2}{e_j}\\
&= 58.76
\end{align*}

Degrees of freedom \\
\begin{align*}
df &= n-1\\
&= 8 - 1\\
&= 7
\end{align*}
If we look to the p-value for the test at 7 df, our value is 58.76 and the $\alpha$ value is far less than 0.001. Since our value is far less than general $\alpha$ interval(0.05, 0.1), we should reject the assumption whether the data follows the uniform distribution.
\subsection*{c}
Since they are both distributions and possibilities, the chi square test can accept both of them simultaneously.


\end{document}
